\documentclass[letterpaper, 10 pt, conference]{IEEEconf}
\usepackage[utf8]{inputenc}
\usepackage[T1]{fontenc}
\usepackage[style=ieee]{biblatex}
\usepackage{amsmath, amssymb, xcolor}

\newcommand{\todo}[1]{{\color{red}#1}}

\title{\LARGE \bf
A Pipeline to Assemble Near-Accident Footage Datasets
}

\author{
         Ben Upenieks, Chaitanya Varier,\\
         Curtis Duy Kha Phan, Jack David Roberts Williamson,\\
         Nicholas Geofroy, Tony Meng, Vincent-Olivier Roch\\
         University of Waterloo\\
         \\
         {\tt\small ben.upenieks@uwaterloo.ca, cvarier@uwaterloo.ca, cdkphan@uwaterloo.ca,
          jdrwilliamson@uwaterloo.ca, nicholas.geofroy@uwaterloo.ca, c24meng@uwaterloo.ca, vroch@uwaterloo.ca}
}

\bibliography{sources}

\begin{document}

\maketitle
\thispagestyle{empty}
\pagestyle{empty}


%%%%%%%%%%%%%%%%%%%%%%%%%%%%%%%%%%%%%%%%%%%%%%%%%%%%%%%%%%%%%%%%%%%%%%%%%%%%%%%%
\begin{abstract}

\todo{Write the abstract}

\end{abstract}

%%%%%%%%%%%%%%%%%%%%%%%%%%%%%%%%%%%%%%%%%%%%%%%%%%%%%%%%%%%%%%%%%%%%%%%%%%%%%%%%
\section{INTRODUCTION}

\todo{Write the introduction} \\

why society needs our tool, what other tools exist and what they are lacking 3/4/5 paragraphs

%%%%%%%%%%%%%%%%%%%%%%%%%%%%%%%%%%%%%%%%%%%%%%%%%%%%%%%%%%%%%%%%%%%%%%%%%%%%%%%%
\section{BACKGROUND / PRELIMINARY}
\todo{write the background / preliminary} \\
introducing basic ideas to readers. why is the annotation pipeline important (how was the annotation toolbox developed) - use separate subsections

%%%%%%%%%%%%%%%%%%%%%%%%%%%%%%%%%%%%%%%%%%%%%%%%%%%%%%%%%%%%%%%%%%%%%%%%%%%%%%%%
\section{METHODOLOGY}

\todo{Write the METHODOLOGY}
\begin{itemize}
  \item Major pipeline
  \item Presenting a graph or figure of the pipeline architecture
  \begin{itemize}
    \item pre-processing
    \item Tracking we can mention how prepare to solve the tracking problem (challenge and future work)
    \item Blurring (tracking and blurring of vehicle plates and faces)
  \end{itemize}
\end{itemize}
\subsection{Video Preprocessing}
\subsection{Automatic Annotation}
\subsection{Object Detection \& Tracking}
\subsection{Blurring}


%%%%%%%%%%%%%%%%%%%%%%%%%%%%%%%%%%%%%%%%%%%%%%%%%%%%%%%%%%%%%%%%%%%%%%%%%%%%%%%%
\section{EXPERIMENT}
- present results of our data, reference the Kitti dataset and how they presented their own data (histogram of object classes, velocity, acceleration, etc)
- the major contribution we have is the histogram of accident type
- Histogram of data
- Usage of our data (applications) - not a major section, just describe potential in preventing/predicting accidents and their importance


\todo{Write the EXPERIMENT SECTION (analyzing data)}


%%%%%%%%%%%%%%%%%%%%%%%%%%%%%%%%%%%%%%%%%%%%%%%%%%%%%%%%%%%%%%%%%%%%%%%%%%%%%%%%
\section{CONCLUSION}

\todo{Write the conclusion}
A conclusion section is not required. Although a conclusion may review the main points of the paper, do not replicate the abstract as the conclusion. A conclusion might elaborate on the importance of the work or suggest applications and extensions.

%%%%%%%%%%%%%%%%%%%%%%%%%%%%%%%%%%%%%%%%%%%%%%%%%%%%%%%%%%%%%%%%%%%%%%%%%%%%%%%%
\section{FUTURE WORK}

\todo{Write the future work ideas}

\addtolength{\textheight}{-12cm}

%%%%%%%%%%%%%%%%%%%%%%%%%%%%%%%%%%%%%%%%%%%%%%%%%%%%%%%%%%%%%%%%%%%%%%%%%%%%%%%%

\section*{APPENDIX}

\todo{Write the appendix}

%%%%%%%%%%%%%%%%%%%%%%%%%%%%%%%%%%%%%%%%%%%%%%%%%%%%%%%%%%%%%%%%%%%%%%%%%%%%%%%%

\nocite{*}
\printbibliography

\end{document}
